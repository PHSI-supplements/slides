\documentclass{article}
\usepackage{amsmath}
% \usepackage{cancel}
\usepackage{fancyhdr}
\usepackage{graphicx}
\usepackage{hyperref}
\usepackage[procnames]{listings}
\lstset{language=C, alsolanguage=[x86masm]Assembler, tabsize=4, upquote=true, basicstyle=\ttfamily}
\usepackage{xcolor}
\newcommand{\function}[1]{\textbf{\lstinline{#1}}}

% \pagestyle{fancy}
% \fancyfoot{}
% \lfoot{\copyright\ Christopher A. Bohn}
% %% (c) 2018-22
% \rfoot{Page \thepage}

\fancypagestyle{plain}{
    \fancyhf{}
    \lfoot{\copyright\ Christopher A. Bohn} % (c) 2018-22
    \rfoot{Page \thepage}
    \renewcommand{\headrulewidth}{0pt}
    \renewcommand{\footrulewidth}{0pt}
}
\pagestyle{plain}

\newcommand{\carry}{\scriptstyle 1}

\begin{document}

\title{Chapter 6 Worked Examples}
\date{}
\maketitle

Any examples that you want to take all the way to assembly code can easily be handled with a compiler on a laptop or on the server (\texttt{gcc -Og -S \dots}).
Alternatively, you might consider using the ``Compiler Explorer'' website (\href{https://godbolt.org/}{https://godbolt.org/}) -- this will generate the assembly code for nearly any language, for many processors, and with many compilers (if the processor is x86-64 and there's enough code to generate an executable, then it will also produce the output).
Another nifty thing is that hovering the mouse over a line of assembly (or C) causes the corresponding C (or assembly) code to highlight.
Select ``C'' as the source language, and unless specified otherwise, select a recent version of ``x86-64 gcc'' or ``x86-64 clang'' as the compiler, with ``-Og'' as the compiler argument.

The code snippets from the textbook are available at \href{https://github.com/PHSI-supplements/textbook-code/tree/main/chap06}{https://github.com/PHSI-supplements/textbook-code/tree/main/chap06}.

\section{A note about registers used in function calls}

While this is covered in more detail when we get to function calls, the textbook examples all occur inside functions (this should be unsurprising).
To understand which registers correspond to what: students would do well to \textbf{remember} that on x86-64, \texttt{\%rdi} is the first argument, \texttt{\%rsi} is the second argument, and \texttt{\%rax} holds the return value.
(Remembering this is also beneficial in the lab assignment. *cough* *cough*)
Some of the examples have three or four arguments -- encourage students to consult Table~6.5 in Chapter~6 to see which registers are used for additional arguments (or, if a function in a lab is reading from a register that doesn't appear to have been written to, it's probably an argument, and students can determine which argument by using that table).

\section{Structured Data}

\subsection{1D Arrays}

\begin{itemize}
\item Contiguous region of memory with homogeneous data
\item Size of the array: $number\_of\_elements \times sizeof(element\_type)$
\item The array variable is a pointer to the 0th element -- the address held by the pointer is the array's \textit{base address}
\item Address of element $i$: $base\_address + i \times sizeof(element\_type)$
    \begin{itemize}
    \item If $i$ is a compile-time constant, gcc will likely use displacement-mode addressing
    \item If $i$ is a variable, use indexed mode $(R_b, R_i, S)$
        \begin{itemize}
            \item $R_b$ is the register that holds the base address (it corresponds to the array variable)
            \item $R_i$ is the register that holds the index variable ($i$)
            \item $S$ is the size of the element type (1, 2, 4, or 8)
        \end{itemize}
    \end{itemize}
\item Demonstrate with \function{get_address} and \function{get_value} in \textit{arrays.c} from the GitHub repo
\item The \function{traverse_array} function in \textit{lecture-examples.c} also demonstrates this
\item The \function{traverse_array_with_pointers} function demonstrates the relationship between array indexing and pointer arithmetic
\end{itemize}

Suppose we have \lstinline{int foo[]}, and \lstinline{foo}'s base address is 0x1400. Assuming that \lstinline{sizeof(int)==4}, what is the address of \lstinline{foo[32]}?

\textit{Answer:} $0x1400 + 4 \times 32_{10} = 0x1400 + 4 \times 0x20 = 0x1480$ \\

Suppose we have \lstinline{long bar[]}, and \lstinline{bar}'s base address is 0x1400. Assuming that \lstinline{sizeof(long)==8}, what is the address of \lstinline{foo[14]}?

\textit{Answer:} $0x1400 + 8 \times 14_{10} = 0x1400 + 16_{10} \times 7 = 0x1400 + 0x10 \times 7 = 0x1470$

\subsection{Nested Arrays}

\begin{itemize}
\item Conceptually, a 2-dimensional array (or $n$-dimensional array) of elements, but memory is 1-dimensional
\item C views a nested array as an ``array of rows'' -- each row is an array of elements
    \begin{itemize}
    \item C stores arrays in row-major order
    \item Knowing that C stores arrays in row-major order is important now for finding an element
    \item Knowing that C stores arrays in row-major order will have performance impacts when we get to Chapter 10 (memory)
    \end{itemize}
\item Can only use nested arrays if number of columns is known (no problem if number of rows is unknown)
\item Size of the array: $number\_of\_rows \times sizeof(row) \\ = number\_of\_rows \times number\_of\_columns \times sizeof(element\_type)$
\item Address of element $i,j$: \\
\begin{align*}
&\phantom{==} base\_address\_of\_row\_i + j \times sizeof(element\_type) \\
&= base\_address + i \times sizeof(row) + j \times sizeof(element\_type) \\
&= base\_address + i \times number\_of\_columns \times sizeof(element\_type) + j \times sizeof(element\_type) \\
&= base\_address + \left( i \times number\_of\_columns + j \right) \times sizeof(element\_type)
\end{align*}
\end{itemize}

Suppose we have \lstinline{int xyzzy[][48]}, and \lstinline{xyzzy}'s base address is 0x1400. Assuming that \lstinline{sizeof(int)==4}, what is the size of a row?

\textit{Answer:} $48_{10} \times 4 = 0x30 \times 4 = 0b11'0000 << 2 = 0b1100'0000 = 0x\textrm{C}0 = 192_{10}$ bytes

What is the address of \lstinline{xyzzy[8]}?

\textit{Answer:} $0x1400 + 0x\textrm{C}0 \times 8 = 0x1400 + 0b1100'0000 << 3 = \\ 0x1400 + 0b110'0000'0000 = 0x1400 + 0x600 = 0x1\mathrm{A}00$

What is the address of \lstinline{xyzzy[8][3]}?

\textit{Answer:} $0x1\mathrm{A}00 + 3 \times 4 = 0x1\mathrm{A}00 + 0xC = 0x1\mathrm{A}0\textrm{C}$ \\

Here's an example that I plan to include in the textbook's second edition, using a variable-length array:
\begin{lstlisting}[language=C]
long foo(long col, int array[][col], long i, long j) {
    return array[i][j];
}
\end{lstlisting}

which produces the assembly code
\begin{lstlisting}[language={[x86masm]Assembler}]
foo:
    imulq   %rdi, %rdx
    leaq    (%rsi,%rdx,4), %rax
    movl    (%rax,%rcx,4), %eax
    ret
\end{lstlisting}

This is very easy to read.
First, $i \times col$ is computed by multiplying the contents of \lstinline{\%rdi} and \lstinline{\%rdx}.
Then that, along with the base address in \lstinline{\%rsi} and the size of an \lstinline{int} (4 bytes) is used to calculate the address of the $i^{th}$ row.
Finally, scaled-indexed addressing mode is used to retrieve the content of the $j^{th}$ element of that row. \\

\function{nested_array1} from \textit{arrays.c}:
\begin{lstlisting}[language=C]
long nested_array1(long i, long j) {
    long bar[5][7];
    long x=bar[i][j];
    return x;
}
\end{lstlisting}
becomes (comments demonstrate the address of element i,j)
\begin{lstlisting}[language={[x86masm]Assembler}]
nested_array1:                      # 5 rows, 7 columns, 8 bytes/long
    subq    $176, %rsp
    leaq    0(,%rdi,8), %rax        # rax = 8*i
    subq    %rdi, %rax              # rax = 8*i-i = 7*i
    addq    %rsi, %rax              # rax = 7*i + j
    movq    -120(%rsp,%rax,8), %rax # base_addr=-120(%rsp), sizeof(long)=8
    addq    $176, %rsp
    ret
\end{lstlisting}
We can see that $i \times number\_of\_columns + j$ = \lstinline{7*i + j}

The code in \textit{2darray.c} can be used to generate problems in which students determine the nested array's dimensions.
Set the value of M \& N (but don't show the students), generate the assembly, and then walk through the assembly code to figure out what M \& N are.

\begin{lstlisting}[language=C]
#define M ...
#define N ...

long A[M][N];
long B[N][M];

long foo(long i, long j) {
    return A[i][j] + B[j][i];
}
\end{lstlisting}

The resulting assembly might be
\begin{lstlisting}[language={[x86masm]Assembler}]
foo:                                # base addresses in A, B
    leaq    (%rsi,%rdi,4), %rdx     # rdx = j + 4*i
    leaq    (%rsi,%rsi,2), %rax     # rax = j + 2*j = 3*j
    leaq    (%rdi,%rax,2), %rax     # rax = i + 2*(3*j) = i + 6*j
    movq    B(,%rax,8), %rax        # retrieves element at B + 8*(6*j + i)
    addq    A(,%rdx,8), %rax        # retrieves element at A + 8*(4*i + j)
    ret
\end{lstlisting}
From this we can see that array A has 4 columns, and array B has 6 columns.
Therefore, $M=6$ and $N=4$.

\subsection{Iliffe Vectors}

\begin{itemize}
\item Array of pointers to rows
\item Used when number of columns is unknown, or if rows don't all have the same number of elements (\textit{i.e.}, a jagged array)
\item Requires two memory accesses to access element
    \begin{itemize}
    \item First index the array of pointers to get the row's base address
    \item Then index the row
    \item Contrast with one memory access to retrieve element in nested array
    \end{itemize}
\item Declared as \lstinline{type **variable} or \lstinline{type *variable[]}
    \begin{itemize}
    \item Contrast with \lstinline{type variable[][]} in nested array
    \item Both Iliffe vectors and nested arrays accessed as \lstinline{variable[row][column]} -- but the resulting assembly code is very different for the two
    \end{itemize}
\end{itemize}

No interesting examples worth discussing -- the array of pointers is a 1D array, and each of the rows is either a 1D array (possibly an Iliffe vector) or a nested array, both of which we've already looked at.

\subsection{Structs (aka Records)}

\begin{itemize}
\item Contiguous region of memory with hetereogenous data
\item Each field is a constant number of bytes after the base address -- use displacement-mode addressing
\item Offset is determined by the sizes of the previous elements and any padding
\item Fields must be appear in the declared order -- compiler is \textit{not} free to find a more-compact representation
\item A datatype that requires $k$ bytes must occur at an address divisible by $k$ -- this is why padding is present
\item The struct must itself have a base address that is divisible by $k_{max}$, where $k_{max}$ is the size if its largest datatype -- this is to guarantee that all fields are properly aligned in memory
\end{itemize}

Consider:

\begin{lstlisting}[language=C]
struct foo {
    int i;
    int j;
    long k;
}
\end{lstlisting}

The \lstinline{i} field is at the structure's base address.
The \lstinline{j} field occurs 4 bytes later, because \lstinline{i} requires 4 bytes, and \lstinline{j} must occur at an address divisible by 4 (so no padding is required).
The \lstinline{k} field occurs 4 bytes later, or 8 bytes after the base address, because \lstinline{i} and \lstinline{j} togetehr require 8 bytes, and \lstinline{k} must occur at an address divisible by 8 (so no padding is required).
The \lstinline{foo} structure requires $4+4+8=16_{10}$ bytes.

Now consider:

\begin{lstlisting}[language=C]
struct bar {
    char c;
    short s;
    short t;
    int i;
    long l;
}
\end{lstlisting}

The \lstinline{c} field is at the structure's base address.
The \lstinline{s} field needs to be at an address divisible by 2, so there's a byte of ``padding'' (unused space) after \lstinline{c} -- this places \lstinline{s} at 2 bytes after the base address.
The \lstinline{t} field is two bytes later, 4 bytes after the base address.
The \lstinline{i} field cannot be 6 bytes after the base address because it needs to be at an address divisible by 4.
By adding 2 bytes of padding after \lstinline{t}, \lstinline{i} is 8 bytes after the base address.
Similarly, there are 4 bytes of padding after \lstinline{i} so that \lstinline{l} is 16 bytes after the base address.
The \lstinline{bar} structure requires 24 byte.

Now consider:

\begin{lstlisting}[language=C]
struct quux {
    int i;
    char c;
    char s[20];
    char *p;
}
\end{lstlisting}

The \lstinline{c} occurs 4 bytes after the base address (because \lstinline{i} occupies 4 bytes).
Is there padding between \lstinline{c} and \lstinline{s}?
No, because \lstinline{s}'s datatype is \lstinline{char}, and so it can start at any address divisible by 1.
So, \lstinline{s} starts 5 bytes after the base address and occupies 20 bytes.
Can \lstinline{p} be 25 bytes after the base address?
No, because \lstinline{p}'s datatype is \lstinline{char*} -- since pointers occupy 8 bytes, \lstinline{p} must be at an address divisible by 8.
That means that there's 7 bytes of padding after \lstinline{s}, putting \lstinline{p} at 32 bytes after the base address.
The \lstinline{quux} structure requires 40 bytes.

Note that this structure demonstrates a scenario in which arrays and pointers are not perfectly synonymous.
Suppose we had this code:

\begin{lstlisting}[language=C]
    struct quux q;
    q.p = malloc(20);
    char t[] = "hello, world!";
    q.p = strcpy(q.p, t);
    printf("%s\n", q.p);
\end{lstlisting}

This code compiles and runs just fine.
The exact form of the second-to-last line is uncommon: there isn't a need to assign \function{strcpy}'s return pointer to \lstinline{q.p}, but there's no harm in doing so.
On the other hand:

\begin{lstlisting}[language=C]
    struct quux q;
    char t[] = "hello, world!";
    q.s = strcpy(q.s, t);
    printf("%s\n", q.s);
\end{lstlisting}

won't compile at all!
You cannot make an assignment to \lstinline{s} because it is actually a piece of a structure, and not a pointer to an arbitrary address.
If we remove the assignment, the code will compile and run without difficulty:

\begin{lstlisting}[language=C]
    struct quux q;
    char t[] = "hello, world!";
    strcpy(q.s, t);
    printf("%s\n", q.s);
\end{lstlisting}

Consider compiling and executing the \textit{struct\_alignment.c} program.
This will print the actual memory addresses of some structures, to demonstrate the presense of padding.
(If you look at the assembly code to see displacement mode, \lstinline{x} is stored at \lstinline{64(%rsp)}, \lstinline{y} is stored at \lstinline{48(%rsp)}, and \lstinline{z} is stored at \lstinline{16(%rsp)}.)

\section{Conditionals}

\section{Loops}

\section{Function Calls}

\begin{tabular}{|l|l||l|l|} \hline
Register & Value & Memory Address & Value \\ \hline
\%rax & 0x04   & 0xF028 & 0x35   \\ \hline
\%rbx & 0x08   & 0xF020 & 0x20   \\ \hline
\%rcx & 0xF010 & 0xF018 & 0x70   \\ \hline
\%rdx & 0x34   & 0xF010 & 0x6294 \\ \hline
\%rsp & 0xF008 & 0xF008 & 0x62A8 \\ \hline
\%rbp & 0xF00C & 0xF000 & 0x35   \\ \hline
\%rip & 0x6283 & \dots  &        \\ \hline
\end{tabular}

(The following examples are cumulative.)

\textit{What address is the top of the stack?} 0x3008

\subsection{Push/Pop}

\textit{What is the effect of \texttt{pushq \%rbx}?} subtract 8 from \%rsp, and copy the content of \%rbx to the top of the stack \\
\begin{tabular}{|l|l||l|l|} \hline
Register & Value & Memory Address & Value \\ \hline
\%rax & 0x04   & 0xF028 & 0x35   \\ \hline
\%rbx & 0x08   & 0xF020 & 0x20   \\ \hline
\%rcx & 0xF010 & 0xF018 & 0x70   \\ \hline
\%rdx & 0x34   & 0xF010 & 0x6294 \\ \hline
\%rsp & \textcolor{red}{0xF000} & 0xF008 & 0x62A8 \\ \hline
\%rbp & 0xF00C & 0xF000 & \textcolor{red}{0x08}   \\ \hline
\%rip & \textcolor{blue}{0x6284} & \dots  &        \\ \hline
\end{tabular} \\
(the instruction pointer also updates by 1 byte because pushq is a 1-byte instruction -- but that's not the part we're interested in for this example) \\

\textit{What is the effect of \texttt{pop \%rax}?} copy the content at the top of the stack into \%rax, and add 8 to the stack pointer \\
\begin{tabular}{|l|l||l|l|} \hline
Register & Value & Memory Address & Value \\ \hline
\%rax & \textcolor{red}{0x08}   & 0xF028 & 0x35   \\ \hline
\%rbx & 0x08   & 0xF020 & 0x20   \\ \hline
\%rcx & 0xF010 & 0xF018 & 0x70   \\ \hline
\%rdx & 0x34   & 0xF010 & 0x6294 \\ \hline
\%rsp & \textcolor{red}{0xF008} & 0xF008 & 0x62A8 \\ \hline
\%rbp & 0xF00C & 0xF000 & 0x08   \\ \hline
\%rip & \textcolor{blue}{0x6285} & \dots  &        \\ \hline
\end{tabular} \\

\textit{What is the effect of \texttt{pop \%rcx}?} copy the content at the top of the stack into \%rcx, and add 8 to the stack pointer \\
\begin{tabular}{|l|l||l|l|} \hline
Register & Value & Memory Address & Value \\ \hline
\%rax & 0x08   & 0xF028 & 0x35   \\ \hline
\%rbx & 0x08   & 0xF020 & 0x20   \\ \hline
\%rcx & \textcolor{red}{0x62A8} & 0xF018 & 0x70   \\ \hline
\%rdx & 0x34   & 0xF010 & 0x6294 \\ \hline
\%rsp & \textcolor{red}{0xF010} & 0xF008 & 0x62A8 \\ \hline
\%rbp & 0xF00C & 0xF000 & 0x08   \\ \hline
\%rip & \textcolor{blue}{0x6286} & \dots  &        \\ \hline
\end{tabular} \\

\textit{What is the effect of \texttt{pushq \%rbp}?} subtract 8 from \%rsp, and copy the content of \%rbp to the top of the stack \\
\begin{tabular}{|l|l||l|l|} \hline
Register & Value & Memory Address & Value \\ \hline
\%rax & 0x08   & 0xF028 & 0x35   \\ \hline
\%rbx & 0x08   & 0xF020 & 0x20   \\ \hline
\%rcx & 0x62A8 & 0xF018 & 0x70   \\ \hline
\%rdx & 0x34   & 0xF010 & 0x6294 \\ \hline
\%rsp & \textcolor{red}{0xF008} & 0xF008 & \textcolor{red}{0xF00C} \\ \hline
\%rbp & 0xF00C & 0xF000 & 0x08   \\ \hline
\%rip & \textcolor{blue}{0x6287} & \dots  &        \\ \hline
\end{tabular} \\

\subsection{Call/Return}

\textit{What is the effect of \texttt{call 0x63A9 <foo>}?} push the return address onto the stack, and jump to the function foo at address 0x63A9

\textit{What is the return address?} the address of the instruction following the call; on x86-64, that instruction will be at \%rip+5

\textit{Obviously \%rsp is changed because we're pushing the return address onto the stack; what other register is changed?} the target address is loaded into \%rip \\
\begin{tabular}{|l|l||l|l|} \hline
Register & Value & Memory Address & Value \\ \hline
\%rax & 0x08   & 0xF028 & 0x35   \\ \hline
\%rbx & 0x08   & 0xF020 & 0x20   \\ \hline
\%rcx & 0x62A8 & 0xF018 & 0x70   \\ \hline
\%rdx & 0x34   & 0xF010 & 0x6294 \\ \hline
\%rsp & \textcolor{red}{0xF000} & 0xF008 & 0xF00C \\ \hline
\%rbp & 0xF00C & 0xF000 & \textcolor{red}{0x628C}   \\ \hline
\%rip & \textcolor{red}{0x63A9} & \dots  &        \\ \hline
\end{tabular} \\

\textit{What is the effect of \texttt{ret}?} pop the return address into the instruction pointer \\
\begin{tabular}{|l|l||l|l|} \hline
Register & Value & Memory Address & Value \\ \hline
\%rax & 0x08   & 0xF028 & 0x35   \\ \hline
\%rbx & 0x08   & 0xF020 & 0x20   \\ \hline
\%rcx & 0x62A8 & 0xF018 & 0x70   \\ \hline
\%rdx & 0x34   & 0xF010 & 0x6294 \\ \hline
\%rsp & \textcolor{red}{0xF008} & 0xF008 & 0xF00C \\ \hline
\%rbp & 0xF00C & 0xF000 & 0x628C   \\ \hline
\%rip & \textcolor{red}{0x628C} & \dots  &        \\ \hline
\end{tabular} \\

Suppose that we execute \texttt{pop \%rbp} to get the stack pointer pointing to 0xF010. \\
\textit{What is the effect of \texttt{ret}?} pop the return address into the instruction pointer \\
\begin{tabular}{|l|l||l|l|} \hline
Register & Value & Memory Address & Value \\ \hline
\%rax & 0x08   & 0xF028 & 0x35   \\ \hline
\%rbx & 0x08   & 0xF020 & 0x20   \\ \hline
\%rcx & 0x62A8 & 0xF018 & 0x70   \\ \hline
\%rdx & 0x34   & 0xF010 & 0x6294 \\ \hline
\%rsp & \textcolor{red}{0xF018} & 0xF008 & 0xF00C \\ \hline
\%rbp & 0xF00C & 0xF000 & 0x628C   \\ \hline
\%rip & \textcolor{red}{0x6294} & \dots  &        \\ \hline
\end{tabular} \\


\end{document}
