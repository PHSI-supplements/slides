\documentclass{article}
\usepackage{amsmath}
\usepackage{cancel}
% \usepackage{bigdelim}
% \usepackage{textcomp}

\newcommand{\carry}{\scriptstyle 1}

\begin{document}

\title{Chapter 3 Worked Examples}
\date{}
\maketitle

\section{Find $-x$}

\[-x = \sim x + 1\]

\subsection{What is the bit pattern for $-5$?}

\subsubsection{Using 4 bits}
\begin{align*}
5_{10} &= 0101_2 \\ \\
-0101  &= \sim 0101 + 1 \\
       &= 1010 + 1 = 1011 \\
\mathrm{checking\dots} & \\
1011_2 &= -1 \times 2^3 + 0 \times 2^2 + 1 \times 2^1 + 1 \times 2^0 \\
       &= -8 + 0 + 2 + 1 = -5_{10}
\end{align*}

\subsubsection{Using 8 bits}
\begin{align*}
5_{10} &= 0000\ 0101_2 \\ \\
-0000\ 0101  &= \sim 0000\ 0101 + 1 \\
       &= 1111\ 1010 + 1 = 1111\ 1011 \\
\mathrm{checking\dots} & \\
1111\ 1011_2 &= -2^7 + 2^6 + 2^5 + 2^4 + 2^3 + 2^1 + 2^0 \\
       &= -128 + 64 + 32 + 16 + 8 + 2 + 1 = -5_{10}
\end{align*}

\subsubsection{Using 16 bits}
\begin{align*}
5_{10} &= 0000\ 0000\ 0000\ 0101_2 \\ \\
-0000\ 0000\ 0000\ 0101  &= \sim 0000\ 0000\ 0000\ 0101 + 1 \\
       &= 1111\ 1111\ 1111\ 1010 + 1 = 1111\ 1111\ 1111\ 1011
\end{align*}

\vspace{1cm}

\subsection{What is the bit pattern for $-0x\mathrm{A7}$ (assuming 8-bit integer)?}

\begin{align*}
-\mathrm{A7}_{16} &= -1010\ 0111_2 \\
                  &= \sim 1010\ 0111 + 1 \\
                  &= 0101\ 1000 + 1 \\
                  &= 0101\ 1001 \\
                  &= 59_{16} \\
\mathrm{checking\dots} & \\
\mathrm{A7}_{16}  &= 1010\ 0111_2 \\
                  &= -128 + 32 + 4 + 2 + 1 = -89_{10} \\
-\mathrm{A7}_{16} &= 59_{16} \\
                  &= 0101\ 1001_2 \\
                  &= 64 + 16 + 8 + 1 = 89_{10} = -(-89)_{10}
\end{align*}


\section{Addition}

\subsection{4-bit examples}

Unsigned values range $0..15$, and signed values range $-8..7$.

\begin{equation*}\begin{array}{cccccl}
  & \carry & \carry & & & \leftarrow\mathrm{carry\ bits}\\
  & 0 & 1 & 1 & 0 & \\
+ & 0 & 0 & 1 & 1 & \\ \cline{1-5}
  & 1 & 0 & 0 & 1 &
\end{array}\end{equation*}

\begin{itemize}
\item As unsigned addition: overflow did not occur (final carry-out (0) is the
    same as the initial carry-in (0)). $6+3=9$
\item As signed addition: overflow did occur (the operands' sign bits match (0)
    but the sum's sign bit is different (1)). $6+3=-7$
    \begin{itemize}
    \item We know that $1001_2$ is $-7_{10}$ because \\
        $-1001_2 = \sim 1001 + 1 = 0110 + 1 = 0111_2 = 7_{10} = -(-7)_{10}$
    \end{itemize}
\end{itemize}

\begin{equation*}\begin{array}{ccccc}
 \carry & \carry & \carry & & \\
  & 0 & 1 & 1 & 0 \\
+ & 1 & 0 & 1 & 1 \\ \hline
  & 0 & 0 & 0 & 1
\end{array}\end{equation*}

\begin{itemize}
\item As unsigned addition: overflow did occur (final carry-out (1) is
    different than the initial carry-in (0)). $6+11=1$
\item As signed addition: overflow did not occur (the operands' sign bits
    differ, so signed overflow is not possible). $6+(-5)=-1$
\end{itemize}

\begin{equation*}\begin{array}{ccccc}
  &   &   &   & \\
  & 0 & 1 & 0 & 0 \\
+ & 0 & 0 & 1 & 0 \\ \hline
  & 0 & 1 & 1 & 0
\end{array}\end{equation*}

\begin{itemize}
\item As unsigned addition: overflow did not occur (final carry-out (0) is the
    same as the initial carry-in (0)). $4+2=6$
\item As signed addition: overflow did not occur (the operands' sign bits
    match (0), and the sum's sign bit is the same (0). $4+2=6$
\end{itemize}

\begin{equation*}\begin{array}{ccccc}
 \carry & \carry & & & \\
  & 1 & 1 & 0 & 0 \\
+ & 1 & 1 & 1 & 0 \\ \hline
  & 1 & 0 & 1 & 0
\end{array}\end{equation*}

\begin{itemize}
\item As unsigned addition: overflow did occur (final carry-out (1) is
    different than the initial carry-in (0)). $12+14=10$
\item As signed addition: overflow did not occur (the operands' sign bits
    match (1), and the sum's sign bit is the same (1)). $(-4)+(-2)=(-6)$
\end{itemize}

\begin{equation*}\begin{array}{ccccc}
 \carry & & & & \\
  & 1 & 0 & 1 & 0 \\
+ & 1 & 1 & 0 & 1 \\ \hline
  & 0 & 1 & 1 & 1
\end{array}\end{equation*}

\begin{itemize}
\item As unsigned addition: overflow did occur (final carry-out (1) is
    different than the initial carry-in (0)). $10+13=7$
\item As signed addition: overflow did occur (the operands' sign bits match (1)
    but the sum's sign bit is different (0)). $(-6)+(-3)=7$
\end{itemize}


\section{Subtraction}

\subsection{4-bit examples}

\begin{equation*}\begin{array}{cccccccrcclcccccc}
  &   &   &   &   &        &   &   &   &   &   &            & \carry & \carry & & & \carry \\
  & 0 & 1 & 0 & 0 &        &   & 0 & 1 & 0 & 0 &            &   & 0 & 1 & 0 & 0 \\
- & 0 & 0 & 1 & 1 & \equiv & + &\sim 0 &0&1& 1 & + 1\ \equiv & + & 1 & 1 & 0 & 0 \\ \cline{1-5}\cline{7-11}\cline{13-17}
  &   &   &   &   &        &   &   &   &   &   &            &   & 0 & 0 & 0 & 1
\end{array}\end{equation*}

\begin{itemize}
\item Notice that we took the bitwise complement of the second operand, and made
    the ``+1'' the initial carry-in -- this is what the hardware does
\item As unsigned addition: overflow did not occur (final carry-out (1) is the
    same as the initial carry-in (1)). $4-3=1$
\item As signed addition: overflow did not occur (the operands' sign bits
    differ, so signed overflow is not possible). $4-3=1$
\item Yes, I intentionally left the phrasing as ``As (un)signed addition''
    because the rules we're using only make work \textit{when the operation is
    viewed as addition}.
\end{itemize}

\begin{equation*}\begin{array}{ccccccccccc}
  &   &   &   &   &        &   &   & \carry & \carry & \carry \\
  & 0 & 0 & 1 & 1 &        &   & 0 & 0 & 1 & 1 \\
- & 0 & 1 & 0 & 0 & \equiv & + & 1 & 0 & 1 & 1 \\ \cline{1-5}\cline{7-11}
  &   &   &   &   &        &   & 1 & 1 & 1 & 1
\end{array}\end{equation*}

\begin{itemize}
\item As unsigned addition: overflow did occur (final carry-out (0) is
    different than the initial carry-in (1)). $3-4=15$
\item As signed addition: overflow did not occur (the operands' sign bits
    differ, so signed overflow is not possible). $3-4=-1$
\end{itemize}

\begin{equation*}\begin{array}{ccccccccccc}
  &   &   &   &   &        &   & \carry & \carry & \carry & \carry \\
  & 0 & 1 & 0 & 1 &        &   & 0 & 1 & 0 & 1 \\
- & 1 & 1 & 0 & 0 & \equiv & + & 0 & 0 & 1 & 1 \\ \cline{1-5}\cline{7-11}
  &   &   &   &   &        &   & 1 & 0 & 0 & 1
\end{array}\end{equation*}

\begin{itemize}
\item As unsigned addition: overflow did occur (final carry-out (0) is
    different than the initial carry-in (1)). $5-12=9$
\item As signed addition: overflow did occur (the operands' sign bits match (0)
    but the sum's sign bit is different (1)). $5-(-4)=-7$
\end{itemize}

\begin{equation*}\begin{array}{ccccccccccc}
  &   &   &   &   &        &   &   & \carry & \carry & \carry \\
  & 0 & 0 & 1 & 1 &        &   & 0 & 0 & 1 & 1 \\
- & 1 & 1 & 0 & 0 & \equiv & + & 0 & 0 & 1 & 1 \\ \cline{1-5}\cline{7-11}
  &   &   &   &   &        &   & 0 & 1 & 1 & 1
\end{array}\end{equation*}

\begin{itemize}
\item As unsigned addition: overflow did occur (final carry-out (0) is
    different than the initial carry-in (1)). $3-12=7$
\item As signed addition: overflow did not occur (the operands' sign bits
    match (0), and the sum's sign bit is the same (0)). $3-(-4)=7$
\end{itemize}

\begin{equation*}\begin{array}{ccccccccccc}
  &   &   &   &   &        & \carry & \carry & \carry &   & \carry \\
  & 1 & 1 & 1 & 0 &        &   & 1 & 1 & 1 & 0 \\
- & 0 & 1 & 0 & 1 & \equiv & + & 1 & 0 & 1 & 0 \\ \cline{1-5}\cline{7-11}
  &   &   &   &   &        &   & 1 & 0 & 0 & 1
\end{array}\end{equation*}

\begin{itemize}
\item As unsigned addition: overflow did not occur (final carry-out (1) is the
    same as the initial carry-in (1)). $14-5=9$
\item As signed addition: overflow did not occur (the operands' sign bits
    match (10), and the sum's sign bit is the same (1)). $(-2)-5=-7$
\end{itemize}


\section{Multiplication}

\subsection{Multiplication by 2}

\begin{align*}
3_{10} \times 2_{10} &= 6_{10} \\
0000\ 0011_2 \times 2_{10} &= 0000\ 0110_2 \\
\mathrm{note\ that} & \\
0000\ 0011_2 << 1 &= 0000\ 0110_2
\end{align*}

\begin{align*}
9_{10} \times 2_{10} &= 18_{10} \\
0000\ 1001_2 \times 2_{10} &= 0001\ 0010_2 \\
\mathrm{note\ that} & \\
0000\ 1001_2 << 1 &= 0001\ 0010_2
\end{align*}

\subsection{Multiplication by 4}

\begin{align*}
3_{10} \times 4_{10} &= 12_{10} \\
0000\ 0011_2 \times 4_{10} &= 0000\ 1100_2 \\
\mathrm{note\ that} & \\
0000\ 0011_2 \times 4_{10} &= 0000\ 0011_2 \times 2_{10} \times 2_{10} \\
                           &= (0000\ 1100_2 << 1) \times 2_{10} \\
                           &= (0000\ 1100_2 << 1) << 1 \\
                           &= 0000\ 0011_2 << 2 \\
                           &= 0000\ 1100_2
\end{align*}

\subsection{Multiplication by $2^n$}

\[ y \times 2^n = y << n \]

\begin{align*}
5_{10} \times 32_{10} &= 0000\ 0101_2 \times 2^5 \\
                      &= 0000\ 0101_2 << 5 \\
                      &= 1010\ 0000_2 \\
                      &= 128 + 32 = 160_{10}
\end{align*}

\subsection{Multiplication by Arbitrary Integer}

\[ y \times b = y \times \sum_{i=0}^{n-1}b_i = \sum_{i=0}^{n-1}\left( y \times b_i \right) \]

\begin{align*}
5_{10} \times 6_{10} &= 0000\ 0101_2 \times 0000\ 0110_2 \\
                     &= 0000\ 0101_2 \times \left( 2^2 + 2^1 \right) \\
                     &= \left( 0000\ 0101_2 \times 2^2 \right) + \left( 0000\ 0101_2 \times 2^1 \right) \\
                     &= (0000\ 0101_2 << 2) + (0000\ 0101_2 << 1) \\
                     &= 0001\ 0100_2 + 0000\ 1010_2 \\
                     &= 0001\ 1110_2 = 30_{10}
\end{align*}

\begin{align*}
9_{10} \times 11_{10} &= 0000\ 1001_2 \times 0000\ 1011_2 \\
                      &= 0000\ 1001_2 \times \left( 2^3 + 2^1 + 2^0 \right) \\
                      &= \left( 0000\ 1001_2 \times 2^3 \right) + \left( 0000\ 1001_2 \times 2^1 \right) + \left( 0000\ 1001_2 \times 2^0 \right) \\
                      &= (0000\ 1001_2 << 3) + (0000\ 1001_2 << 1) + (0000\ 1001_2 << 0) \\
                      &= 0100\ 1000_2 + 0001\ 0010_2 + 0000\ 1001_2 \\
                      &= 0110\ 0011_2 = 99_{10}
\end{align*}

\begin{align*}
12_{10} \times 36_{10} &= 0000\ 1100_2 \times 0010\ 0100_2 \\
                       &= 0000\ 1100_2 \times \left( 2^5 + 2^2 \right) \\
                       &= \left( 0000\ 1100_2 \times 2^5 \right) + \left( 0000\ 1100_2 \times 2^2 \right) \\
                       &= (0000\ 1100_2 << 5) + (0000\ 1100_2 << 2) \\
                       &= 1\ 1000\ 0000_2 + 0011\ 0000_2 \\
                       &= 1\ 1011\ 0000_2 = 432_{10} \\
\mathrm{however,\ if\ we\ have\ only\ 8\ bits} & \\
                       &= \xcancel{1}\ 1011\ 0000_2 = 176_{10}\ \mathrm{(unsigned)} \\
\end{align*}


\section{Division}

\[ y \div 2^n = y >> n \]

\begin{align*}
96_{10} \div 16_{10} &= 0110\ 0000_2 \div 2^4 \\
                     &= 0110\ 0000_2 >> 4 \\
                     &+ 0000\ 0110_2 = 6_{10}
\end{align*}

\begin{align*}
28_{10} \div 8_{10} &= 0001\ 1100_2 \div 2^3 \\
                    &= 0001\ 1100_2 >> 3 \\
                    &= 0000\ 0011.\xcancel{1}_2 = 3_{10}
\end{align*}

\end{document}
