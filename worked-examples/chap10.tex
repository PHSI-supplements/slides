\documentclass{article}
\usepackage{amsmath}
% \usepackage{cancel}
\usepackage{fancyhdr}
\usepackage{multirow}

% \pagestyle{fancy}
% \fancyfoot{}
% \lfoot{\copyright\ Christopher A. Bohn}
% %% (c) 2018-22
% \rfoot{Page \thepage}

\fancypagestyle{plain}{
    \fancyhf{}
    \lfoot{\copyright\ Christopher A. Bohn} % (c) 2018-22
    \rfoot{Page \thepage}
    \renewcommand{\headrulewidth}{0pt}
    \renewcommand{\footrulewidth}{0pt}
}
\pagestyle{plain}

\newcommand{\carry}{\scriptstyle 1}

\begin{document}

\title{Chapter 10 Worked Examples}
\author{}
\date{}
\maketitle

\section{Cache Hits \& Misses}

\subsection{Direct-Mapped Cache}

A notional system has direct-mapped cache that is addressed via 16-bit addresses. The cache uses 8-byte blocks and has 8 total entries.

\begin{tabular}{||c|c|c||c|c|c|c|c|c|c|c||} \hline\hline
\multicolumn{11}{||c||}{Direct-Mapped Cache; 8-byte blocks, 8 blocks} \\ \hline
Index & Tag & Valid & Byte 0 & Byte 1 & Byte 2 & Byte 3 & Byte 4 & Byte 5 & Byte 6 & Byte 7 \\ \hline\hline
0 & 1DA & 1 & 01 & 23 & 45 & 67 & 89 & AB & CD & EF \\ \hline
1 & 345 & 1 & F1 & E2 & D3 & C4 & B5 & A6 & 97 & 80 \\ \hline
2 & 2B7 & 0 & -- & -- & -- & -- & -- & -- & -- & -- \\ \hline
3 & 09C & 1 & 17 & 39 & 28 & 13 & 46 & 79 & 55 & 00 \\ \hline
4 & 18E & 0 & -- & -- & -- & -- & -- & -- & -- & -- \\ \hline
5 & 2F6 & 1 & 11 & 22 & 33 & 44 & 55 & 66 & 77 & 88 \\ \hline
6 & 3A4 & 1 & 99 & 00 & FF & EE & DD & CC & BB & AA \\ \hline
7 & 06B & 1 & 02 & 31 & 06 & 85 & 88 & 00 & 00 & 00 \\ \hline\hline
\end{tabular}

How many bits for the block offset? \underline{3}

How many bits for the set index? \underline{3}

How  many bits for the tag? \underline{10}

\phantom{x}

\begin{tabular}{llcrlcrlcr}
& 15 & $\cdots$ & 6 & 5 & $\cdots$ & 3 & 2 & $\cdots$ & 0 \\ \cline{2-10}
address & \multicolumn{3}{|c|}{tag} & \multicolumn{3}{|c|}{set index} & \multicolumn{3}{|c|}{block offset} \\ \cline{2-10}
\end{tabular}

\phantom{x}

For each of the examples, write address in binary, separate it into the three
bitfields, and then re-express those bitfields in hexadecimal to use with the
cache tables.

\begin{enumerate}

\item Given an address of 0x7687 \\ 0b 0111 0110 1000 0111

    \begin{tabular}{rccc}
    0b & 01 1101 1010 & 000 & 111 \\ 0x & 1DA & 0 & 7 \\
    \end{tabular}

    \begin{itemize}
    \item What is the block offset? \underline{7}
    \item What is the set index? \underline{0}
    \item What is the tag? \underline{0x1DA}
    \item Do we have a cache hit? If so, what is the value of the byte at that address? \underline{yes, 0xEF}
    \end{itemize}

\item Given an address of 0x1D4E
    \begin{itemize}
    \item What is the block offset? \underline{6}
    \item What is the set index? \underline{1}
    \item What is the tag? \underline{0x075}
    \item Do we have a cache hit? If so, what is the value of the byte at that address? \underline{no}
    \end{itemize}

\item Given an address of 0x7215
    \begin{itemize}
    \item What is the block offset? \underline{5}
    \item What is the set index? \underline{2}
    \item What is the tag? \underline{0x1C8}
    \item Do we have a cache hit? If so, what is the value of the byte at that address? \underline{no}
    \end{itemize}

\item Given an address of 0x271C
    \begin{itemize}
    \item What is the block offset? \underline{4}
    \item What is the set index? \underline{3}
    \item What is the tag? \underline{0x09C}
    \item Do we have a cache hit? If so, what is the value of the byte at that address? \underline{yes, 0x46}
    \end{itemize}

\item Given an address of 0x3FA3
    \begin{itemize}
    \item What is the block offset? \underline{3}
    \item What is the set index? \underline{4}
    \item What is the tag? \underline{0x0FE}
    \item Do we have a cache hit? If so, what is the value of the byte at that address? \underline{no}
    \end{itemize}

\item Given an address of 0xBDAA
    \begin{itemize}
    \item What is the block offset? \underline{2}
    \item What is the set index? \underline{5}
    \item What is the tag? \underline{0x2F6}
    \item Do we have a cache hit? If so, what is the value of the byte at that address? \underline{yes, 0x33}
    \end{itemize}

\item Given an address of 0xE931
    \begin{itemize}
    \item What is the block offset? \underline{1}
    \item What is the set index? \underline{6}
    \item What is the tag? \underline{0x3A4}
    \item Do we have a cache hit? If so, what is the value of the byte at that address? \underline{yes, 0x00}
    \end{itemize}

\item Given an address of 0xD638
    \begin{itemize}
    \item What is the block offset? \underline{0}
    \item What is the set index? \underline{7}
    \item What is the tag? \underline{358}
    \item Do we have a cache hit? If so, what is the value of the byte at that address? \underline{no}
    \end{itemize}

\end{enumerate}



\subsection{2-Way Set-Associative Cache}

A notional system has a 2-way set associative cahce that is addressed via 16-bit addresses. The cache uses 8-byte blocks and has 8 total entries.

\begin{tabular}{||c|c|c||c|c|c|c|c|c|c|c||} \hline\hline
\multicolumn{11}{||c||}{2-Way Set-Associative Cache; 8-byte blocks, 8 blocks} \\ \hline
Index & Tag & Valid & Byte 0 & Byte 1 & Byte 2 & Byte 3 & Byte 4 & Byte 5 & Byte 6 & Byte 7 \\ \hline\hline
\multirow{2}{*}{0} & 123 & 1 & 01 & 23 & 45 & 67 & 89 & AB & CD & EF \\ \cline{2-11}
                   & 4A5 & 1 & F1 & E2 & D3 & C4 & B5 & A6 & 97 & 80 \\ \hline
\multirow{2}{*}{1} & 6B7 & 0 & -- & -- & -- & -- & -- & -- & -- & -- \\ \cline{2-11}
                   & 69C & 1 & 17 & 39 & 28 & 13 & 46 & 79 & 55 & 00 \\ \hline
\multirow{2}{*}{2} & 0FE & 0 & -- & -- & -- & -- & -- & -- & -- & -- \\ \cline{2-11}
                   & 1D2 & 1 & 11 & 22 & 33 & 44 & 55 & 66 & 77 & 88 \\ \hline
\multirow{2}{*}{3} & 3A4 & 1 & 99 & 00 & FF & EE & DD & CC & BB & AA \\ \cline{2-11}
                   & 58B & 1 & 02 & 31 & 06 & 85 & 88 & 00 & 00 & 00 \\ \hline\hline
\end{tabular}

How many bits for the block offset? \underline{3}

How many bits for the set index? \underline{2}

How many bits for the tag? \underline{11}

\phantom{x}

\begin{tabular}{llcrlrlcr}
& 15 & $\cdots$ & 5 & 4 & 3 & 2 & $\cdots$ & 0 \\ \cline{2-9}
address & \multicolumn{3}{|c|}{tag} & \multicolumn{2}{|c|}{set index} & \multicolumn{3}{|c|}{block offset} \\ \cline{2-9}
\end{tabular}

\phantom{x}

For each of the examples, write address in binary, separate it into the three
bitfields, and then re-express those bitfields in hexadecimal to use with the
cache tables.

\begin{enumerate}

\item Given an address of 0x2460 \\ 0b 0010 0100 0110 0000

    \begin{tabular}{rccc}
    0b & 001 0010 0011 & 00 & 000 \\ 0x & 123 & 0 & 0 \\
    \end{tabular}

    \begin{itemize}
    \item What is the block offset? \underline{0}
    \item What is the set index? \underline{0}
    \item What is the tag? \underline{0x123}
    \item Do we have a cache hit? If so, what is the value of the byte at that address? \underline{yes, 0x01}
    \end{itemize}

\item Given an address of 0x98A1
    \begin{itemize}
    \item What is the block offset? \underline{1}
    \item What is the set index? \underline{0}
    \item What is the tag? \underline{0x4C5}
    \item Do we have a cache hit? If so, what is the value of the byte at that address? \underline{no}
    \end{itemize}

\item Given an address of 0xD6EA
    \begin{itemize}
    \item What is the block offset? \underline{2}
    \item What is the set index? \underline{1}
    \item What is the tag? \underline{0x6B7}
    \item Do we have a cache hit? If so, what is the value of the byte at that address? \underline{no}
    \end{itemize}

\item Given an address of 0x396B
    \begin{itemize}
    \item What is the block offset? \underline{3}
    \item What is the set index? \underline{1}
    \item What is the tag? \underline{0x1CB}
    \item Do we have a cache hit? If so, what is the value of the byte at that address? \underline{no}
    \end{itemize}

\item Given an address of 0x7F14
    \begin{itemize}
    \item What is the block offset? \underline{4}
    \item What is the set index? \underline{2}
    \item What is the tag? \underline{0x3F8}
    \item Do we have a cache hit? If so, what is the value of the byte at that address? \underline{no}
    \end{itemize}

\item Given an address of 0x3A55
    \begin{itemize}
    \item What is the block offset? \underline{5}
    \item What is the set index? \underline{2}
    \item What is the tag? \underline{0x1D2}
    \item Do we have a cache hit? If so, what is the value of the byte at that address? \underline{yes, 0x66}
    \end{itemize}

\item Given an address of 0xB17E
    \begin{itemize}
    \item What is the block offset? \underline{6}
    \item What is the set index? \underline{3}
    \item What is the tag? \underline{0x58B}
    \item Do we have a cache hit? If so, what is the value of the byte at that address? \underline{yes, 0x00}
    \end{itemize}

\item Given an address of 0x749F
    \begin{itemize}
    \item What is the block offset? \underline{7}
    \item What is the set index? \underline{3}
    \item What is the tag? \underline{0x3A4}
    \item Do we have a cache hit? If so, what is the value of the byte at that address? \underline{yes, 0xAA}
    \end{itemize}

\end{enumerate}




\newpage

\section*{Copy of notional caches for overhead projector or for handing out}

A notional system has direct-mapped cache that is addressed via 16-bit addresses. The cache uses 8-byte blocks and has 8 total entries.

\begin{tabular}{||c|c|c||c|c|c|c|c|c|c|c||} \hline\hline
\multicolumn{11}{||c||}{Direct-Mapped Cache; 8-byte blocks, 8 blocks} \\ \hline
Index & Tag & Valid & Byte 0 & Byte 1 & Byte 2 & Byte 3 & Byte 4 & Byte 5 & Byte 6 & Byte 7 \\ \hline\hline
0 & 1DA & 1 & 01 & 23 & 45 & 67 & 89 & AB & CD & EF \\ \hline
1 & 345 & 1 & F1 & E2 & D3 & C4 & B5 & A6 & 97 & 80 \\ \hline
2 & 2B7 & 0 & -- & -- & -- & -- & -- & -- & -- & -- \\ \hline
3 & 09C & 1 & 17 & 39 & 28 & 13 & 46 & 79 & 55 & 00 \\ \hline
4 & 18E & 0 & -- & -- & -- & -- & -- & -- & -- & -- \\ \hline
5 & 2F6 & 1 & 11 & 22 & 33 & 44 & 55 & 66 & 77 & 88 \\ \hline
6 & 3A4 & 1 & 99 & 00 & FF & EE & DD & CC & BB & AA \\ \hline
7 & 06B & 1 & 02 & 31 & 06 & 85 & 88 & 00 & 00 & 00 \\ \hline\hline
\end{tabular}

\vspace{0.5cm}

\phantom{x} \\
A notional system has a 2-way set associative cahce that is addressed via 16-bit addresses. The cache uses 8-byte blocks and has 8 total entries.

\begin{tabular}{||c|c|c||c|c|c|c|c|c|c|c||} \hline\hline
\multicolumn{11}{||c||}{2-Way Set-Associative Cache; 8-byte blocks, 8 blocks} \\ \hline
Index & Tag & Valid & Byte 0 & Byte 1 & Byte 2 & Byte 3 & Byte 4 & Byte 5 & Byte 6 & Byte 7 \\ \hline\hline
\multirow{2}{*}{0} & 123 & 1 & 01 & 23 & 45 & 67 & 89 & AB & CD & EF \\ \cline{2-11}
                   & 4A5 & 1 & F1 & E2 & D3 & C4 & B5 & A6 & 97 & 80 \\ \hline
\multirow{2}{*}{1} & 6B7 & 0 & -- & -- & -- & -- & -- & -- & -- & -- \\ \cline{2-11}
                   & 69C & 1 & 17 & 39 & 28 & 13 & 46 & 79 & 55 & 00 \\ \hline
\multirow{2}{*}{2} & 0FE & 0 & -- & -- & -- & -- & -- & -- & -- & -- \\ \cline{2-11}
                   & 1D2 & 1 & 11 & 22 & 33 & 44 & 55 & 66 & 77 & 88 \\ \hline
\multirow{2}{*}{3} & 3A4 & 1 & 99 & 00 & FF & EE & DD & CC & BB & AA \\ \cline{2-11}
                   & 58B & 1 & 02 & 31 & 06 & 85 & 88 & 00 & 00 & 00 \\ \hline\hline
\end{tabular}

\end{document}
