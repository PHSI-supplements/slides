\documentclass{article}
\usepackage{amsmath}
\usepackage{bigdelim}
\usepackage{fancyhdr}

% \pagestyle{fancy}
% \fancyfoot{}
% \lfoot{\copyright\ Christopher A. Bohn}
% %% (c) 2018-22
% \rfoot{Page \thepage}

\fancypagestyle{plain}{
    \fancyhf{}
    \lfoot{\copyright\ Christopher A. Bohn} % (c) 2018-22
    \rfoot{Page \thepage}
    \renewcommand{\headrulewidth}{0pt}
    \renewcommand{\footrulewidth}{0pt}
}
\pagestyle{plain}

\begin{document}

\title{Chapter 2 Worked Examples}
\date{}
\maketitle

\section{ASCII Table}

See Table~2.2 in the textbook, or \texttt{man ascii} to view the ASCII table.
Some things to note:
\begin{itemize}
\item 7 bits per ASCII character (a \texttt{char} uses 8 bits to round-up to an
    integral number of bytes (1 byte))
\item Includes printable and non-printable characters
\item Within the range of decimal numerals, the numerals are in-order; within
    the range of uppercase letters, the letters are in alphabetical order;
    within the range of lowercase letters, the letters are in alphabetical order
    \begin{itemize}
    \item Thus, `I' $<$ `J'
    \item But `i' $>$ `J'
    \end{itemize}
\item Notice that we used integer comparisons -- \texttt{char} is an integer
    type! Anything that you can do to an \texttt{int}, you can do to a
    \texttt{char}
    \begin{itemize}
    \item Compare, add, subtract, index an array\dots
    \end{itemize}
\end{itemize}

\section{Number Bases}

Note: in C, if a literal begins with 1-9, it is a decimal literal (base 10). If
a literal begins with 0 followed by 0-9, it is an octal literal (base 8); we
will not use octal. If a literal begins with 0x, it is a hexadecimal literal
(base 16). As a gcc extension, if a literal begins with 0b, it is a binary
literal.\footnote{The draft C23 standard makes binary literals starting with 0b
part of the standard.}

\subsection{Weighted Sums}

\begin{tabular}{rrl}
Base 10 & $3798_{10}$           & $3 \times 10^3 + 7 \times 10^2 + 9 \times 10^1 + 8 \times 10^0$ \\
Base 16 & $\mathrm{ED6}_{16}$   & $14 \times 16^2 + 13 \times 16^1 + 6 \times 16^0$ \\
Base 2  & $1110\ 1101\ 0110_2$  & $1 \times 2^{11} + 1 \times 2^{10} + 1 \times 2^9 + 0 \times 2^8 + \dots + 1 \times 2^1 + 0 \times 2^0$
\end{tabular}

\subsection{Conversion Between Radices}

\subsubsection{Binary to Decimal}

Apply weighted sum technique:

\begin{align*}
1110\ 1101\ 0110_2 &= 2^{11} &&+ 2^{10} &&+ 2^9 &&+ 2^7 &&+ 2^6 &&+ 2^4 &&+ 2^2 &&+ 2^1 \\
    &= 2048 &&+ 1024 &&+ 512 &&+ 128 &&+ 64 &&+ 16 &&+ 4 &&+ 2 \\
    &= 3798_{10}
\end{align*}

\begin{align*}
1001\ 1111\ 0111_2 &= 2^{11} &&+ 2^8 &&+ 2^7 &&+ 2^6 &&+ 2^5 &&+ 2^4 &&+ 2^2 &&+ 2^1 &&+ 2^0\\
    &= 2048 &&+ 256 &&+ 128 &&+ 64 &&+ 32 &&+ 16 &&+ 4 &&+ 2 &&+ 1 \\
    &= 2551_{10}
\end{align*}

\subsubsection{Decimal to Binary}

\textbf{Repeated Division}

Divide by 2; the remainder of each step becomes a bit, starting at the
least-significant bit:

\begin{align*}
3798_{10} \div 2    &= 1899\ r\ 0\ (ones\ place) \\
1899 \div 2         &= 949\ r\ 1\ (twos\ place) \\
949 \div 2          &= 474\ r\ 1\ (fours\ place) \\
474 \div 2          &= 237\ r\ 0\ (eights\ place) \\
237 \div 2          &= 118\ r\ 1\ (sixteens\ place) \\
118 \div 2          &= 59\ r\ 0\ (thirty-twos\ place) \\
59 \div 2           &= 29\ r\ 1\ (sixty-fours\ place) \\
29 \div 2           &= 14\ r\ 1\ (128s\ place) \\
14 \div 2           &= 7\ r\ 0\ (256s\ place) \\
7 \div 2            &= 3\ r\ 1\ (512s\ place) \\
3 \div 2            &= 1\ r\ 1\ (1024s\ place) \\
1 \div 2            &= 0\ r\ \underline{1\ (2048s\ place)} \\
                    &\hspace{1cm}1110\ 1101\ 0110_2
\end{align*}

\begin{align*}
2551_{10} \div 2    &= 1275\ r\ 1\ (ones\ place) \\
1275 \div 2         &= 637\ r\ 1\ (twos\ place) \\
736 \div 2          &= 318\ r\ 1\ (fours\ place) \\
318 \div 2          &= 159\ r\ 0\ (eights\ place) \\
159 \div 2          &= 79\ r\ 1\ (sixteens\ place) \\
79 \div 2           &= 39\ r\ 1\ (thirty-twos\ place) \\
39 \div 2           &= 19\ r\ 1\ (sixty-fours\ place) \\
19 \div 2           &= 9\ r\ 1\ (128s\ place) \\
9 \div 2            &= 4\ r\ 1\ (256s\ place) \\
4 \div 2            &= 2\ r\ 0\ (512s\ place) \\
2 \div 2            &= 1\ r\ 0\ (1024s\ place) \\
1 \div 2            &= 0\ r\ \underline{1\ (2048s\ place)} \\
                    &\hspace{1cm}1001\ 1111\ 0111_2
\end{align*}

\textbf{Repeated Subtraction}

\begin{table}
\centering
\begin{tabular}{|c|r|c|c|r|}\cline{1-2}\cline{4-5}
$2^0$ & 1 & \phantom{extra space} & $2^{10}$ & 1024 \\ \cline{1-2}\cline{4-5}
$2^1$ &   2 & & $2^{11}$ &    2048 \\ \cline{1-2}\cline{4-5}
$2^2$ &   4 & & $2^{12}$ &    4096 \\ \cline{1-2}\cline{4-5}
$2^3$ &   8 & & $2^{13}$ &    8192 \\ \cline{1-2}\cline{4-5}
$2^4$ &  16 & & $2^{14}$ &  16,384 \\ \cline{1-2}\cline{4-5}
$2^5$ &  32 & & $2^{15}$ &  32,768 \\ \cline{1-2}\cline{4-5}
$2^6$ &  64 & & $2^{16}$ &  65,536 \\ \cline{1-2}\cline{4-5}
$2^7$ & 128 & & $2^{17}$ & 131,072 \\ \cline{1-2}\cline{4-5}
$2^8$ & 256 & & $2^{18}$ & 262,144 \\ \cline{1-2}\cline{4-5}
$2^9$ & 512 & & $2^{19}$ & 524,288 \\ \cline{1-2}\cline{4-5}
\end{tabular}
\caption{First 20 Powers of Two}
\end{table}

Subtract by powers-of-two; if subtracting by the next-greatest power-of-two
yields a non-negative difference then subtract and make the next bit (starting
at the most-significant bit) a 1; otherwise don't subtract and make the next bit
a 0. This technique is useful for relatively-small values, as it requires
knowing powers-of-two.\footnote{Personally, I have 1..8192 memorized; after
that it's ``sixteen-thousand-and-change,'' ($2^{14}$)
``thirty-two-thousand-and-change'' ($2^{15}$),
``sixty-five-thousand-and-change'' ($2^{16}$), and ``four-million-and-change''
($2^{32}$).}

\begin{align*}
3798_{10} - 2048    &= 1750 &\rightarrow& 1\ (2048s\ place) \\
1750 - 1024         &= 726  &\rightarrow& 1\ (1024s\ place) \\
726 - 512           &= 214  &\rightarrow& 1\ (512s\ place) \\
214 - 256           &< 0    &\rightarrow& 0\ (256s\ place) \\
214 - 128           &= 86   &\rightarrow& 1\ (128s\ place) \\
86 - 64             &= 22   &\rightarrow& 1\ (sixty-fours\ place) \\
22 - 32             &< 0    &\rightarrow& 0\ (thirty-twos\ place) \\
22 - 16             &= 6    &\rightarrow& 1\ (sixteens\ place) \\
6 - 8               &< 0    &\rightarrow& 0\ (eights\ place) \\
6 - 4               &= 2    &\rightarrow& 1\ (fours\ place) \\
2 - 2               &= 0    &\rightarrow& 1\ (twos\ place) \\
0 - 1               &< 0    &\rightarrow& \underline{0\ (ones\ place)} \\
                    &       &           & 1110\ 1101\ 0110_2
\end{align*}

\begin{align*}
2551_{10} - 2048    &= 503  &\rightarrow& 1\ (2048s\ place) \\
503 - 1024          &< 0    &\rightarrow& 0\ (1024s\ place) \\
503 - 512           &< 0    &\rightarrow& 0\ (512s\ place) \\
503 - 256           &= 247  &\rightarrow& 1\ (256s\ place) \\
247 - 128           &= 119  &\rightarrow& 1\ (128s\ place) \\
119 - 64            &= 55   &\rightarrow& 1\ (sixty-fours\ place) \\
55 - 32             &= 23   &\rightarrow& 1\ (thirty-twos\ place) \\
23 - 16             &= 7    &\rightarrow& 1\ (sixteens\ place) \\
7 - 8               &< 0    &\rightarrow& 0\ (eights\ place) \\
7 - 4               &= 3    &\rightarrow& 1\ (fours\ place) \\
3 - 2               &= 1    &\rightarrow& 1\ (twos\ place) \\
1 - 1               &= 0    &\rightarrow& \underline{1\ (ones\ place)} \\
                    &       &           & 1001\ 1111\ 0111_2
\end{align*}

\subsubsection{Binary to Hexadecimal, and Hexadecimal to Binary}

\begin{table}
\centering
\begin{tabular}{|c|c|c|l}\cline{1-3}
Hexadecimal & Binary    & Decimal   & \\ \cline{1-3}
0           & 0000      & 0         & \rdelim \}{10}{*}[BCD range] \\
1           & 0001      & 1         & \\
2           & 0010      & 2         & \\
3           & 0011      & 3         & \\
4           & 0100      & 4         & \\
5           & 0101      & 5         & \\
6           & 0110      & 6         & \\
7           & 0111      & 7         & \\
8           & 1000      & 8         & \\
9           & 1001      & 9         & \\ \cline{3-3}
A           & 1010      & 10        & \\
B           & 1011      & 11        & \\
C           & 1100      & 12        & \\
D           & 1101      & 13        & \\
E           & 1110      & 14        & \\
F           & 1111      & 15        & \\ \cline{1-3}
\end{tabular}
\caption{Binary/Hexadecimal Conversion Table}\label{tab:conversion}
\end{table}

Because each hex-digit corresponds to exactly four bits, no math is necessary.
Simply (re-create and) use the binary/hex conversion table
(Table~\ref{tab:conversion}). For example:

\begin{tabbing}
$1110$\ \= $1101$\ \= $0110_2$\ \= \\
$\downarrow$ \> $\downarrow$ \> $\downarrow$ \> \\
E \> D \> 6 \> = $\mathrm{ED6}_{16}$
\end{tabbing}

\begin{tabbing}
$\mathrm{9F7}_{16}$ = \= 9\phantom{000} \= F\phantom{000} \= 7\phantom{000} \\
\> $\downarrow$ \> $\downarrow$ \> $\downarrow$ \\
\> $1110$\ \> $1101$\ \> $0111_2$
\end{tabbing}

\subsubsection{Decimal to Hexadecimal, and Hexadecimal to Decimal}

We \textit{could} apply the weighted-sum and repeated-division techniques as
before:

\begin{align*}
\mathrm{9F7}_{16} &= 9 \times 16^2 &&+ 15 \times 16^1 &&+ 7 \times 16^0 \\
    &= 9 \times 256 &&+ 15 \times 16 &&+ 7 \times 1 \\
    &= 2304 &&+ 240 &&+ 7\\
    &= 2551_{10}
\end{align*}

\begin{align*}
3798_{10} \div 16   &= 237\ r\ 6 &\rightarrow& 6\ (ones\ place) \\
159 \div 16         &= 14\ r\ 13 &\rightarrow& D\ (sixteens\ place) \\
14 \div 16          &= 0\ r\ 14 &\rightarrow& \underline{E\ (256s\ place)} \\
                    &&& \mathrm{ED6}_{16}
\end{align*}

But it's probably easier to use binary as an intermediate form since we already
know how to convert between binary \& decimal and between binary \& hexadecimal:

\[\mathrm{9F7}_{16} \leftrightarrow 1001\ 1111\ 0111_2 \leftrightarrow 2551_{10}\]

\[3798_{10} \leftrightarrow 1110\ 1101\ 0110_2 \leftrightarrow \mathrm{ED6}_{16}\]

\end{document}
